\documentclass[a4paper,12pt]{article}

\usepackage{titlesec}
\titleformat*{\section}{\large\bfseries}
\titleformat*{\subsection}{\normalsize\bfseries}

\pagestyle{empty}

%\usepackage[ngerman]{babel}

\usepackage[utf8]{inputenc}
\usepackage[top=0.8in, bottom=0.8in, left=0.8in, right=0.8in]{geometry}

\usepackage{graphicx}
\usepackage{url}
\usepackage{paralist}
\usepackage{lmodern}
\usepackage[authoryear]{natbib}
\usepackage{blindtext}
\renewcommand\textbullet{\ensuremath{\bullet}}

\renewcommand{\familydefault}{\sfdefault}

\date{}

\title{
%\vspace*{-2cm} \includegraphics[width=16cm]{header.png}
\includegraphics[width=6cm]{figures/stuttgart-vector.pdf}\hfill{\includegraphics[width=3cm]{figures/rss_logo.pdf}}
\quad \\ [0.5cm]
{\large \textit{Student Report (Bachelor Software Engineering):}} \\ [1mm]
{\Large Comparison of application monitoring and alerting tools}
}

\begin{document}
	

\maketitle

\thispagestyle{empty}

\vspace{-2.5cm}


\subsection*{Background and Motivation}
Nowerdays big server architectures with spreaded servers are very common. To monitor this big architektures, control tools for very distributed systems are needet. Some of these tools are very modular and can be used only for collecting data of the server or for a web representation of this data. On the over hand some systems offer a complete package with consistentsy in technolgy.\cite{Heger2017}
This Student Report is to evaluate the diverent tools on the market and compare them to print out there main features and disadvantages.

\subsection*{Goals}
The Student Report lists and discusses the different open source monitoring and alerings tools. Goal is to print out the different architekures, features and technologies of the systems to make it easyer for the reader to deside wich aplication is the best for his system.
Some of the compared systems are InfluxDB, Prometheus and ELK. As a first step more tools for comaprison will be addet to the list .
As a part of the report the evaluated systems will be installed on a server to test them in a prakticable eviroment.

\subsection*{Possible Collaborations}


\begin{scriptsize}
\bibliographystyle{abbrv}
\bibliography{bibliography.bib}
\end{scriptsize}

\subsection*{Contact}
Dr.-Ing. André van Hoorn, van.hoorn@informatik.uni-stuttgart.de \\
University of Stuttgart, Inst.\ for Software Technology, Reliable Software Systems Group \\

\end{document}

