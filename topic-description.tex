\documentclass[a4paper,12pt]{article}

\usepackage{titlesec}
\titleformat*{\section}{\large\bfseries}
\titleformat*{\subsection}{\normalsize\bfseries}

\pagestyle{empty}

%\usepackage[ngerman]{babel}

\usepackage[utf8]{inputenc}
\usepackage[top=0.8in, bottom=0.8in, left=0.8in, right=0.8in]{geometry}

\usepackage{graphicx}
\usepackage{url}
\usepackage{paralist}
\usepackage{lmodern}
\usepackage[authoryear]{natbib}
\usepackage{blindtext}
\renewcommand\textbullet{\ensuremath{\bullet}}

\renewcommand{\familydefault}{\sfdefault}

\date{}

\title{
%\vspace*{-2cm} \includegraphics[width=16cm]{header.png}
\includegraphics[width=6cm]{figures/stuttgart-vector.pdf}\hfill{\includegraphics[width=3cm]{figures/rss_logo.pdf}}
\quad \\ [0.5cm]
{\large \textit{Student Report (Bachelor Software Engineering):}} \\ [1mm]
{\Large Comparison of application monitoring and alerting tools}
}

\begin{document}
	

\maketitle

\thispagestyle{empty}

\vspace{-2.5cm}


\subsection*{Background and Motivation}
Nowerdays big server architectures with spreaded Servers are very common. To monitor this big architektures, control tools for very distributed systems are needet. Some of these tools are very Modular and can be user only be used for collecting the data of the server or for a Web Representation of the data. On the over hand some System offer a consistentsy in technolgy by being isolated.\cite{Heger2017}

\subsection*{Goals}
The Student Report lists and discusses the different open Source Monitoring and alerings tools. Goal is to print out the different architekures and technologie of the Systems to make it easyer for the reader to deside wich aplication is the best for his system.
Some of the presented Systems are InfluxDB, Prometheus and ELK.

\subsection*{Possible Collaborations}
Lorem ipsum dolor sit amet, consectetuer adipiscing elit. Etiam lobortis facilisis sem. Nullam nec
mi et neque pharetra sollicitudin. Praesent imperdiet mi nec ante. Donec ullamcorper, felis non
sodales commodo, lectus velit ultrices augue, a dignissim nibh lectus placerat pede. Vivamus nunc
nunc, molestie ut, ultricies vel, semper in, velit. 

\begin{scriptsize}
\bibliographystyle{abbrv}
\bibliography{bibliography.bib}
\end{scriptsize}

\subsection*{Contact}
Dr.-Ing. André van Hoorn, van.hoorn@informatik.uni-stuttgart.de \\
University of Stuttgart, Inst.\ for Software Technology, Reliable Software Systems Group \\

\end{document}

